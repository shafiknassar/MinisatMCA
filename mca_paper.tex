\documentclass[]{article}

\usepackage{fullpage}
\usepackage[utf8]{inputenc}
\usepackage{algorithmic}
\usepackage{amsfonts}
\usepackage{graphicx}
\usepackage{amsthm}
\usepackage{mathrsfs}
\usepackage{amssymb}
\usepackage{hyperref}
\usepackage{amsmath}
\usepackage{cite}
\usepackage{xcolor}
\usepackage{soul}
\usepackage{algorithm2e}
\hypersetup{%
	pdfborder = {0 0 0}
}
\newtheorem{theorem}{Theorem}

\newcommand{\msat}[0]{$MiniSat$ }
%\newtheorem{lemma}{Lemma}
%\newtheorem{definition}{Definition}
%\newtheorem{claim}{Claim}
%\newtheorem{corollary}{Corollary}
%\newtheorem{observation}{Observation}
%\newtheorem{remark}{Remark}
%\newtheorem{oq}{Open Question}
%opening
\title{Rotation in Minimising Conflicting Assumptions}
\author{Shafik Nassar, Pierre Glianos}

\begin{document}
	
	\maketitle
	
	\begin{abstract}
		
		The \textit{Boolean Satisfiability Problem} (SAT), archetypical NP-Complete, is the problem of determining whether a given boolean formula, usually \textit{CNF}, has a satisfying \textit{assignment}. In addition to the theoretical interest SAT draws, it has many modern-day applications in different fields, most notably hardware verification, with SAT-solvers playing an integral part in Bounded Checking Model, for example.\\
		Modern uses require solving SAT instances under given \textit{assumptions}, i.e. partial assignment to the formula's variables. Furthermore, if the instance is \textit{unsatisfiable} under the assumptions, it is interesting to compute a \textit{Minimal set of Conflicting Assumptions} (MCA). For example, computing MCAs \textit{can} be used \textit{indirectly}\footnote{this paper will discuss how, theoretically, the problem of extracting Minimally Unsatisfiable Sub-formulas (MUS) can be reduced to the problem of computing MCAs.  MUS are used in a number of verification tasks to extract a concise description of inconsistency \cite{MUSRMR}} in some verification tasks to extract a concise description of inconsistency.\\
		Any algorithm that computes minimal sets of conflicting assumptions, makes several SAT-solver calls under different assumptions. In this paper, we present algorithms that aim to reduce the number of SAT-solver calls by gathering as much information as possible per call.
		The goal of our project was to design and implement efficient\footnote{with respect to number of SAT-solver calls} algorithms for computing MCAs, compare them and assess their performance. We use a well known open source SAT-solver called MiniSAT. This paper presents the knowledge we needed to acquire in order to approach the problem, the theory and the implementations of the algorithms. \\
		The main algorithm presented in this paper (Iterative Deletion-Based Rotation) is heavily influenced by similar MUS Extraction algorithm called Recursive Model Rotation. Recursive Model Rotation led to multiple orders of magnitude performance improvements on practical instances\footnote{presented in \cite{MUSRMR}}. Mainly for this reason, MUS will be discussed in this paper. To the best of our knowledge, this is the first algorithm for computing Minimal sets of Conflicting Assumptions that uses the technique of Model Rotation.
		
		\paragraph{title}
		
	\end{abstract}
	
	\pagebreak
	\tableofcontents
	\pagebreak
	
	\section{Introduction}
	\textbf{Symbols} \\
	Boolean Satisfiability Problem: SAT \\
	Conjunctive Normal Form (formula): CNF \\
	Computer Science: CS \\
	A CNF formula: $ \mathcal{F} $ (sometimes will be treated as a set of clauses) \\
	A set of variables: $ \mathcal{V} $ \\
	An assignment: $ \phi $ \\
	A set of assumption: $ \mathcal{A} $ \\
	Set of literals in $ \mathcal{V} $: $ lit(\mathcal{V}) $
	\subsection{Introduction to SAT}
	\paragraph{Definitions}
	In the world of boolean formulas, a \textit{variable} can be assigned one of 2 possible values: $ True, False $. A \textit{literal} is an instance of a variable or it's negation, i.e. if $ x $ is a variable, then $ x $ and $ \bar{x} $ are the 2 possible literals of $ x $. Given a set of variables $ \mathcal{V} $, let $ lit(\mathcal{V}) $ be the set of all possible literals in $ \mathcal{V} $. A \textit{clause} is disjunction (OR) of literals and a formula is said to be in \textit{Conjunctive Normal Form} (CNF) if it's a conjunction (AND) of clauses. For example: $\mathcal{F} = (x_1 \lor \bar{x_2}) \land (x_2 \lor x_3) $ is a CNF, and $ \mathcal{V}=\{x_1,x_2,x_3\} $ is the group of variables (notice that $ \mathcal{V} $ doesn't include $ \bar{x_2} $). \\ An \textit{assignment} for the variables is a function $ \phi : \mathcal{V} \rightarrow \{True, False\} $. $ \phi $ \textit{satisfies} $ \mathcal{F} $ if $ \mathcal{F} $ is evaluated to $ True $ under the assignment $ \phi $. If a formula has a satisfying assignment, it is said to be \textit{satisfiable}. If no such assignment exists, then it is said to be \textit{unsatisfiable}. \\
	Now we can define SAT. The Boolean Satisfiability Problem is the problem of determining whether a given boolean formula, usually CNF, is satisfiable. Formally, given $ \mathcal{F} $, determine whether it has a model (denoted $ \mathcal{F} \in SAT $) or not (denoted $ \mathcal{F} \in UNSAT $))
	\paragraph{In Theory} SAT is interesting because it was the first problem proven to be \textit{NP-Complete}, in 1971. Ever since, the most straight forward way to prove a given problem $ P $ is NP-Complete was by showing a \textit{reduction} from SAT to $ P $. Of course, being an NP-Complete problem, finding a polynomial algorithm for solving SAT would mean that $ P=NP $, closing the most famous problem in CS. Since it's widely believed that $ P \neq NP $, the approach to solving such NP-Complete problems in real life applications is simply to try to avoid them as much as possible, since, in most cases, solving the problem would require exponential time. \\
	This was a very important motivation when designing the algorithms presented in this paper; though SAT couldn't be completely avoided in this context, it was desirable to try to reduce, as much as possible, the number of times in which an algorithm is required to solve a SAT instance.
	\pagebreak
	\subsection{Introduction to SAT-Solving}
	\paragraph{History}
	\begin{itemize}
		\item 1960 The first SAT solving algorithm DP ( Davis, Putnam), originally for checking the validity of a first-order logic formula
		\item 1962 An improved version of DP – DPLL, the basis for almost all modern SAT-solvers
		\item 1971 SAT is proved to be the first NP-Complete problem, an important milestone to CS theory
		\item 1992 Local Search SAT solving
		\item 1992 The First International SAT Competition, followed by 1993, 1996, since 2002 every year
		\item 1996 Conflict Driven Clause Learning
		\item 1996 The First International SAT Conference (Workshop), followed by 1998, since 2000 every year
	\end{itemize}
	
	\paragraph{SAT Solving applications}
	\begin{itemize}
		\item Hardware Model Checking - all major hardware companies use SAT solvers to verify their design, additionally there are many companies which specializes in developing SAT solving tools to be used in hardware verification.
		\item Software Verification - used for verifying embedded systems in cars, airplanes, refrigerators etc. usually not over-complexed software.
		\item Automated Planning and Scheduling in Artificial Intelligence - still one of the best approaches for optimal planning.
	\end{itemize}
	
	As mentioned before, SAT is a NP-Complete problem, and therefore all the known complete algorithms have exponential runtime in the worst case. As showed before it has different uses in many fields, thus optimizing the solvers for this problem has a direct affect on the technological advancement of our world. The basic techniques for SAT solving are discussed next.

	\paragraph{The Resolution Rule} Given the two clauses $ (I \lor x_1 \lor x_2 \lor ... \lor x_n)  \land (\bar{I} \lor y_1 \lor y_2 \lor ... \lor y_m)$ the \textit{Resolvent} clause is $ (x_1 \lor x_2 \lor ... \lor x_n \lor y_1 \lor y_2 \lor ... \lor y_m) $. \\
	Special cases:
	\begin{itemize}
		\item Tautological Resolvent: $ R((x_1 \lor x_2), (\bar{x_1} \lor \bar{x_2})) = True $
		\item Empty clause: $ R((x_1), (\bar{x_1})) = \bot $
		\item Unit Resolution: $ R((x_1), (\bar{x_1} \lor x_2 \lor x_3)) = (x_2 \lor x_3)$
	\end{itemize}
	
	\begin{theorem}
		Resolution maintains satisfiability\\
		Given $ \mathcal{F} $ and $ \mathcal{C}_1, \mathcal{C}_2 $ two of it's clauses with a pair complementary literals, then $ \mathcal{F} $ is satisfiable if and only if $ \mathcal{F} \land R(\mathcal{C}_1, \mathcal{C}_2) $ is satisfiable.
	\end{theorem}
	Based on $\textbf{Theorem 1}$, if the empty clause $\bot$ is resolved, the original formula is unsatisfiable. So, the Resolution rule may be used to proof the unsatisfiability of a formula. The following algorithm does exactly that.
	\pagebreak
	
	\begin{algorithm}[H]
		\KwIn{CNF formula $ \mathcal{F} $}
		\KwOut{SAT, UNSTAT}
		\While{true}{
			R $\leftarrow$ resolveAll($ \mathcal{F} $)\\
			\eIf{R $\cap$ $ \mathcal{F} \neq$ R } {
				 $\mathcal{F} \leftarrow \mathcal{F} \cup R $
			}{
				break
			}
		}
		\eIf{$\bot$ $\in$ $\mathcal{F}$}{
			return SAT
		}{
			return UNSAT
		}
	\caption{Saturation Algorithm}
	\end{algorithm}
	The algorithm is sound and complete, but it has exponential time and space complexity - even for "simple" formulas - so it's not really applicable in practice. Nonetheless, we will now demonstrate a very important procedure which makes use of the resolution rule and is applied in many modern solver.
	\paragraph{Unit Propagation} The procedure is based on unit clauses, i.e. a clauses which contains one literal. The resulting formula is equivalent to the original one.\\
	$\forall \mathcal{C} \in \mathcal{F} $ s.t. $ \mathcal{C} $ is a unit clause containing the literal $l$ we apply the following rules:
	\begin{enumerate}
		\item For each clause $\mathcal{C'} $ s.t. $ \mathcal{C'} \neq \mathcal{C} \land l \in \mathcal{C'} $, $\mathcal{C'} $ is removed from $\mathcal{F} $.
		\item For each clause $\mathcal{C'} $ s.t. $\bar{l} \in \mathcal{C'} $, $\bar{l}$ is removed from $\mathcal{C'} $.
	\end{enumerate}
	Since every clause must be satisfied, we know that for unit clauses the literal composing it must be true, thus we can remove every clause containing this literal (except for the unit clause), this is the motivation for the first rule. The second rule is based on unit resolution and the fact that $ (x_1 \lor x_2) \vDash (x_1 \lor x_2 \lor x_3) $.
	
	\paragraph{Pure Literal Deletion} If a literal $ l $ never occurs negated in $ \mathcal{F} $ then each clause containing $ l $ may be deleted from the formula. This is correct because each clause containing $ l $ can be satisfied by setting $ l $ to True without sacrificing other clauses.
	
	\subsubsection{DPLL}
	Davis-Putnam-Logemann-Loveland algorithm was introduced in 1962 and is based on the earlier Davis-Putnam algorithm, though it contains few algorithmic improvements. \\
	The basic idea is simplification and case splitting. Simplification is achieved by applying unit propagation and pure literal deletion to the formula. In case splitting a literal $ l $ is chosen \textit{wisely}\cite{BranchingH} then two recursive calls is performed with $ l $ being set to $ True $ and $ False $ respectively. The algorithm stops when $ \bot $ is implied, there are no clauses left - meaning all literals are assigned - or after all assignments have been tried.
	\\
	\begin{algorithm}[H]
		\KwIn{CNF formula $ \mathcal{F} $}
		\KwOut{$true$, $false$}
		\While{$ \mathcal{F} $ contains unit clause \{$ l $\}} {
			$ \mathcal{F} \leftarrow propogate(\mathcal{F}, l) $
		}
		\If{$\bot \in \mathcal{F} $}{ return $false$ }
		\While{$ \mathcal{F} $ contains pure literal $ l $} {
			$ \mathcal{F} \leftarrow delete\_literal(\mathcal{F}, l) $
		}
		$ l \leftarrow choose\_literal\_for\_splitting(\mathcal{F}) $ \\
		\eIf{$ l = nil $}{return $true$}{return DPLL($\mathcal{F} \cup \{l\})$ or $ DPLL(\mathcal{F} \cup \{\neg l\}) $}
	\caption{DPLL Algorithm}
	\end{algorithm}
	Since it was published, the algorithm has been a hot topic in research and SAT solving competitions. In recent years the work on improving the algorithm has been done on alternative directions, from defining new data structures in order to perform different operations faster (unit propagation), to trying variant policies for choosing the case splitting literal.\\ The most significant improvements are non-chronological backtracking (backjumping) and clause learning. These refinements describe a method of backtracking and learning when reaching a conflict clause. The result are state of the art Conflict-Driven Clause Learning (CDCL) SAT solvers.

	\subsubsection{CDCL}
	
	\pagebreak
	\section{Defining the problem}
	\subsection{Minimal set of Conflicting Assumptions}
	In this section, we define a new problem, one that uses the definitions presented in section 1.1.
	\paragraph{Definitions} A \textit{partial assignment} for $ \mathcal{V} $ is a partial function in $ \mathcal{V} $, i.e. a function $ \phi' : \mathcal{V}' \rightarrow \{True, False\} $ s.t. $\mathcal{V}' \subsetneq \mathcal{V}$. A \textit{set of assumptions} $ \mathcal{A} $ is a partial assignment. $ \mathcal{A} $ can be thought of as a subset of literals in $\mathcal{V}$ that doesn't include a literal and it's negation, formally $ \mathcal{A} \subsetneq lit(\mathcal{V}) $ s.t. $ \forall a \in \mathcal{A} : \bar{a} \notin \mathcal{A} $.\\
	$ \mathcal{F} $ is said to be \textit{satisfiable under the assumptions} $ \mathcal{A} $ if there exists a satisfying assignment $ \phi $ that agrees with $ \mathcal{A} $ (the partial function $ \phi' $ that defines $ \mathcal{A} $ can be extended to the full function $ \phi $). Intuitively, an assumption is some sort of a constraint. The more assumptions we have, the harder it is to find a satisfying assignment.
	\paragraph{Problem definition: MCA} Given $ \mathcal{F} $ and $ \mathcal{A} $ s.t. $ \mathcal{F} $ is unsatisfiable under $ \mathcal{A} $, the \textit{Minimal Conflicting Assumptions} (MCA) of $ \mathcal{A} $ is a subset $ \mathcal{A}' \subseteq \mathcal{A} $ s.t. $ \mathcal{F} $ is unsatisfiable under $ \mathcal{A}' $ and $ \forall a \in \mathcal{A}' $ : $ \mathcal{F} $ is satisfiable under $ \mathcal{A}' \setminus \{a\} $, denoted $ \mathcal{A}' \in MCA(\mathcal{F},\mathcal{A}) $. Intuitively, we want to know which assumptions caused the unsatisfiability (conflict), but we want a minimal set of assumptions that did so. The set is minimal in the sense that removing any assumption from it will eliminate the unsatisfiability.\\
	\paragraph{Example 1 } given  $\mathcal{F} = (x_1 \lor x_2) \land (\bar{x_2} \lor x_3)$ and $\mathcal{A} =\{\bar{x_1}, \bar{x_3}\}$. We can see that $ \mathcal{F} $ is unsatisfiable under $ \mathcal{A} $. $ \mathcal{A} $ is the MCA of itself, since removing $\bar{x_1}$ or $\bar{x_3}$ from it will eliminate the unsatisfiability (for example, the assignment $ \phi(x_1)=True,\phi(x_2)=False, \phi(x_3)=False $ extends $ \mathcal{A} \setminus \{\bar{x_1}\} $ and satisfies $\mathcal{F}$ - similar argument can be used to show that $\bar{x_3}$ is also necessary for the unsatisfiability). This example demonstrates that the MCA can be equal to the original set of assumptions.
	\paragraph{Example 2 } given  $\mathcal{F} = (x_1 \lor x_2 \lor x_5) \land (x_3 \lor x_5) \land (\bar{x_5} \lor \bar{x_4})$ and $\mathcal{A} =\{\bar{x_1}, \bar{x_2}, \bar{x_3}, x_4\}$. Again, we can see that $ \mathcal{F} $ is unsatisfiable under $ \mathcal{A} $. $ x_4 $ implies $ \bar{x_5} $ but $\bar{x_1}$ and $ \bar{x_2}$ imply $ x_5 $. That's a conflict, so $\mathcal{A}_1 =\{\bar{x_1}, \bar{x_2}, x_4\}$ is an MCA. But also $ x_3 $ by itself implies $ x_5 $, so it can replace both $ \bar{x_1}, \bar{x_2} $ in $ \mathcal{A}_1 $ and we get a new MCA, $\mathcal{A}_2 =\{\bar{x_3}, x_4\}$. 
	This example demonstrates that the MCA is not necessarily unique. Furthermore, it demonstrates that "size doesn't matter", i.e. the number of elements in the MCA doesn't affect its status as an MCA.
	
	\paragraph{Next Step} The main focus of the rest of this paper will be designing and implementing efficient algorithms that, given $\mathcal{F}, \mathcal{A}$, compute an MCA.
	
	\subsection[Similar Problem]{Minimizing Conflicting Clauses}
	\paragraph{Definitions} Here, $ \mathcal{F} $ is treated as a set of clauses. $ \mathcal{F} $ is \textit{minimally unsatisfiable} if $ \mathcal{F} \in UNSAT $ and for any $ \mathcal{F}' \subsetneq \mathcal{F} $, $ \mathcal{F}' \in SAT$. Given a formula $ \mathcal{F} \in UNSAT $, a \textit{Minimally Unsatisfiable Sub-formula} of $ \mathcal{F} $ is any $ \mathcal{F}' \subseteq \mathcal{F} $ s.t. $ \mathcal{F}' $ is minimally unsatisfiable, denoted $ \mathcal{F}' \in MUS(\mathcal{F}) $.

	\paragraph{Problem definition: MUS} Given $ \mathcal{F} $, compute any $ \mathcal{F}' \in MUS(\mathcal{F}) $.
	
	\paragraph{Connection to MCA} The definition of MUS is very similar to MCA. In both problems, given an unsatisfiable instance, we search for a minimal (in respect to set inclusion) subset of objects that is responsible for the unsatisfiability. We will in fact prove that MUS Extraction can be reduced to Computing MCA.
	
	\begin{theorem}
		MUS Extraction can be reduced to Computing MCA
	\end{theorem}
	\begin{proof} \textit{Scheme}
		Given $ \mathcal{G}=\{c_1,c_2,...,c_m\} \in UNSAT$ with $ m $ clauses, we will construct a pair  $ <\mathcal{F}=\{c'_1,c'_2,...,c'_m\},\mathcal{A}=\{a_1,a_2,...,a_m\}> $. Intuitively, each original clause $ c_i \in \mathcal{G} $ will have an indicator $ a_i \in \mathcal{A} $; if the indicator is turned on ($ a_i$ is assumed to be True) then the assignment must satisfy $ c_i $, else, the assignment may or may not satisfy $ c_i $. We'd argue that for every subset $\{n_i\}_1^k \subseteq [m] $, if we denote  ${G}'=\{c_{n_i}\}_1^k \subseteq \mathcal{G}, \mathcal{A}'=\{a_{n_i}\}_1^k \subseteq \mathcal{A}$ then the following logical biconditional holds: \\ 
		$ \mathcal{G}'\in MUS(\mathcal{G}) \iff  \mathcal{A}'\in MCA(\mathcal{F}, \mathcal{A}) $ \\
		Let $ \{a_1,...,a_m\} $ be a set of variables disjoint to $ Var(\mathcal{G}) $. The pair will be constructed as follows:
		\begin{itemize}
			\item $ \mathcal{F}=\{c'_i | c_i = (x_1 \lor ... \lor x_{k_i})\in\mathcal{G}, c'_i= (x_1 \lor ... \lor x_{k_i} \lor \bar{a_i}) \} $
			\item $ \mathcal{A}=\{a_1,...,a_m\} $
		\end{itemize}
	$ \mathcal{F} $ is UNSAT under $ \mathcal{A} $ since none of the newly added literals $ \bar{a_1},...,\bar{a_m} $ can satisfy any clause (because $ \mathcal{A} $ implies, by definition, $ \{a_1,...,a_m\} $, which evaluate every $ \bar{a_i} $ to False), and therefore we go back to satisfying $ \mathcal{G}$. But we assume $ \mathcal{G} \in UNSAT $. \\
	Now we prove the logical biconditional:
	\begin{itemize}
		\item Given a subset $ \mathcal{G}'$ s.t. $ {G}' \in MUS(\mathcal{G}) $, then no assignment satisfies $ \mathcal{G}' $, so no assignment satisfies $ \mathcal{F} $ under the corresponding subset of assumptions $ \mathcal{A}'$. Also, for each $ c_i \in \mathcal{G}'$, there exists an assignment $ \phi_i $ that satisfies $ \mathcal{G}' \setminus \{c_i\}$. Let us define a new assignment $ \Phi_i $ that gives every original variable $ x \in Var(\mathcal{G}) $ the value $ \phi_i(x) $, every $ a \in \mathcal{A}'\{a_i\}$ the value $ True $ and every other indicator $ False $. $ \Phi_i $ satisfies $ \mathcal{F} $ under $\mathcal{A}'\{a_i\}$. Therefore, $ \mathcal{A}' \in MCA(\mathcal{F,A}) $.
		\item Given a subset $ \mathcal{A}'$ s.t. $ \mathcal{A}' \in MCA(\mathcal{F,A}) $, a very similar argument can show that the corresponding sub-formula $ {G}' $ is in $ MUS(\mathcal{G})$.
	\end{itemize}
	\end{proof}

	\paragraph{Exploiting Similarities} In section 3, we will demonstrate an algorithm, empirically proven \cite{MUSRMR} to be efficient on practical instances, for MUS Extraction. We will then present a new algorithm for computing MCAs that uses the same technique used by the efficient MUS extractor.

	\pagebreak
	\section{Assumption Minimiser}
	\subsection{Overview}
		The Assumption Minimiser (a.k.a. \textit{mca program}) is an extension to \textit{MiniSat}\footnote{Minisat in an open source SAT solver, it was introduced in 2003 to provide people with a simple and efficient solver. For more info visit \url{http://minisat.se}.}. It can be used in two ways:
		\begin{enumerate}
				\item As an Incremental Minimiser. The program has a basic API which enables it to be used repeatedly as with incremental SAT solvers. The basic actions are described in the next section.
				\item As CLI tool. The program can be invoked from the command line providing the CNF formula, assumptions and the algorithm desired for minimizing. It can be also invoked with the original MiniSat parameters without using the new feature.
		\end{enumerate}
	
	\subsection{API}
	\begin{center}
		\centerline{  \textbf{\textit{ AssumMinimiser API} } }
		\fbox{
			\parbox[b]{10.5cm}{
					$ AssumMinimiser\left( MinisatSolver\& \ solver\right) $ \\ \\
					$ AssumMinimiser\left( MinisatSolver\& \ solver, Vector\& \ assumptions\right) $ \\ \\
					$ void \quad addAssumption\left(Literal \ assumption \right) $ \\ \\
					$ bool \quad removeAssumption\left(Literal \ assumption \right) $ \\ \\
					$ void \quad addClause(Vetor\& \ literals) $ \\ \\
					$ bool \quad isSatWithAssumptions\left(\right) $\\ \\
					$ bool \quad isSatWithoutAssumptions\left(\right) $\\ \\
					$ void \quad iterativeInsertion\left(Vector\& \ result\right) $ \\ \\
					$ void \quad iterativeDeletion\left(Vector\& \ result\right) $ \\ \\
					$ void \quad iterativeDeletion2\left(Vector\& \ result\right) $ \\ \\ 
					$ void \quad rotationAlgorithm\left(Vector\& \ result\right) $ \\ \\
					$ void \quad printStats\left(\right) $ 
			}
		}
	\end{center}

   	\textbf{API description}
    \begin{itemize}
    	\item \textbf{Constructors:} As mentioned before, the minimiser uses $MiniSat$'s solver and data structure. The solver itself should be provided by the user, whereas the minimiser will use the same instance provided. This grants the user direct access to the solver for any actions desired; although it's not recommended to interfere with the solver while one of the minimising algorithms is active. It's also possible to provide a set of initial assumptions, that will be minimised later using one of the algorithms.    	
    	
    	\item \textbf{addAssumption(Literal):} This can be used if the user wishes to extend the set of assumptions
    	
    	\item \textbf{removeAssumption(Literal):} This is a contra-action to the previous method. If the assumption was removed then $  true $ will be returned, else it'll return $ false $.
    	
    	\item \textbf{addClause(Vector):} The method adds the given clause to the solver. One can alternately call the solver's $ addClause $ method directly, achieving the same result.
    	
    	\item \textbf{isSatWithAssumptions():} The method returns $ true $ iff the given formula (inside the solver) is satisfiable under the current set of assumptions.
    	
    	\item \textbf{isSatWithoutAssumptions():} The method returns $ true $ iff the given formula (inside the solver) is satisfiable regardless to the current set of assumptions.
    	
    	\item \textbf{printStats():} The method will print on the screen all the statistics from the last activation of the minimization algorithms.
    	
    	\item \textbf{Minimization algorithms:} This is the core of the program, each call to one of the four algorithms will activate the minimization process, this process might take a while, as we have multiple calls to the SAT solver. The $ mca $ will be returned in the vector that was provided. Further detail on the algorithms in the next section.
    \end{itemize}

	
	\pagebreak
	\section{Basic Algorithms}
	In this section, the basic algorithms in the program are introduced. They're inspired from algorithms for MUS extraction\cite{PracticalAlgs}, and are all local search based. The basic algorithms are the Iterative Insertion and Deletion,  the rest of them are deletion based which contains some optimization.
	\subsection{Iterative Insertion}
		\begin{algorithm}[H]
		\KwIn{Unsatisfiable CNF formula $ \mathcal{F} $ under set of assumptions $ \mathcal{A} $}
		\KwOut{MCA $ \mathcal{M} $ }
		$ \mathcal{M} $ $\leftarrow$ $\emptyset$\\
		\While{solve($ \mathcal{F} $, $ \mathcal{M} $) $\neq$ SAT} {
			$  S  \leftarrow  \mathcal{M} $\\
			\ForEach{$ a \in \mathcal{A} \backslash \mathcal{M} $}{
				$ S \leftarrow  S \cup \{a\} $\\
				\If{solve($ \mathcal{F}$, $ S $) = UNSAT}{
					$ \mathcal{M} $ $\leftarrow$ $ \mathcal{M} \cup \{a\} $ \\
					break
				}
			}
		}
		return $ \mathcal{M} $
		\caption{Iterative Insertion}
	\end{algorithm}
	The algorithm tries to compose the minimal set from scratch. It starts off with an empty set of (vital) assumptions and will stop the search when the formula is unsatisfiable under the  assumptions. In each step a temporary set containing the current vital assumptions is initialized, then it begins to scan the rest of the assumptions, adding them one by one to the temporary set. If the formula is found to be unsatisfiable during this process, the last assumption scanned is added to the minimal set, then the algorithm jumps back to checking the satisfiability of the formula under the vital set.
	
	\paragraph{Correctness} Let $ \mathcal{F} $ be a CNF formula, $ \mathcal{A} $ a set of assumptions s.t. $ \mathcal{F} $ is UNSAT under $ \mathcal{A} $. Let $ S \subseteq \mathcal{A} $, $ a \in \mathcal{A} \backslash S $ s.t. $ \mathcal{F} $ is SAT under $ S $ and UNSAT under $ S \cup \{a\} $. Then there exists an MCA that includes $ a $. Since the algorithm composes the minimal set according to the above rule and halts at the moment the formula is unsatisfiable under this set, guarantees that the result is minimal.
 
	\paragraph{Complexity} As can be noticed, there are two nested loops where each one might perform $ \mathcal{O \left( | M |\right) } $ iterations, so the worst case complexity is $ O \left( | M |^\text{2} \right) $ solver calls.
	
	
	\subsection{Iterative Deletion}
		\begin{algorithm}[H]
			\KwIn{Unsatisfiable CNF formula $ \mathcal{F} $ under set of assumptions $ \mathcal{A} $}
			\KwOut{MCA $ \mathcal{M} $ }
			$ \mathcal{M} $ $\leftarrow$ $ \mathcal{A} $\\
				\ForEach{$ a \in \mathcal{A} $}{
					 $\mathcal{M} \leftarrow \mathcal{M} \backslash \{a\}$\\
					\If{solve($ \mathcal{F} $, $ \mathcal{M} $) = SAT}{
						$ \mathcal{M} \leftarrow \mathcal{M} \cup \{a\} $
					}
				}
			return $ \mathcal{M} $
			\caption{Iterative Deletion}
		\end{algorithm}
		In contrary to the previous algorithm, this one begins with the vital set containing all the assumptions. Then it tries to remove each one at a time, if the formula is satisfiable under the set now then the assumption is vital and is restored to the set.
		
		\paragraph{Correctness} Let $ \mathcal{F} $ be a CNF formula, $ \mathcal{A} $ a set of conflicting assumptions and $  a \in \mathcal{A} $. If $ \mathcal{F} $ is UNSAT under $ \mathcal{A} \backslash \{a\} $ then there exists a MCA that doesn't include $ \{a\} $. If $ \mathcal{F} $ is SAT under $ \mathcal{A} \backslash \{a\} $ then $ \{a\} $ is included in every MCA. As the algorithm filters the non-vital assumptions, the result is assured to be minimal.
		
		\paragraph{Complexity} The algorithm scans the set of assumptions once then the worst case complexity is $ O \left( | M | \right)  $ solver calls.
		
		\subsubsection{Iterative Deletion with Outside Help}
			\begin{algorithm}[H]
				\KwIn{Unsatisfiable CNF formula $ \mathcal{F} $ under set of assumptions $ \mathcal{A} $}
				\KwOut{MCA $ \mathcal{M} $ }
				$ \mathcal{M} $ $\leftarrow$ $ \mathcal{A} $\\
					\ForEach{a $\in \mathcal{A} $}{
						$\mathcal{M} \leftarrow \mathcal{M} \backslash \{a\}$\\
						\eIf{solve($ \mathcal{F} $, $ \mathcal{M} $) = SAT}{
							$ \mathcal{M} $ $\leftarrow$ $ \mathcal{M} $ $\cup$ \{a\} \\
						} {
							\hl{ $ \mathcal{M} $ $\leftarrow$ $ \mathcal{M} $ $\cap$ $ get\_minisat\_conflict() $ \\}
						}
					}
				return $ \mathcal{M} $
				\caption{Iterative Deletion With Outside Help}
			\end{algorithm}
		
		\msat has the ability to solve the problem under assumptions, and as discussed in the first chapter, it has an internal conflict handling mechanism. So, in case the problem is unsatisfiable, \msat can provide the conflicting assumptions.
		
		\paragraph{Correctness} Same as the previous algorithm. The call to \msat in case the formula is unsatisfiable will only remove non conflicting assumptions from the set.
		
		\paragraph{Complexity} Also $ O \left( | M |\right) $ solver calls, as the information that's used from the solver is gathered during the $ solve $ method. In addition, set operations can be done in constant time.
		
	\section{Rotation}
	\subsection{Rotation in Conflicting Clauses}
	\subsection{Borrowing the Idea}
	
	\section{Benchmarks and Results}
	
	\section{Summary}
	
	\pagebreak
	\begin{thebibliography}{9}
		
		\bibitem{MUSRMR}
			Joao Marques-Silva,
			Anton Belov.
			\textit{Accelerating MUS Extraction with Recursive Model Rotation},
			2011.
			
		\bibitem{BranchingH}
			Joao Marques-Silva.
			\textit{The Impact of Branching Heuristics in Propositional Satisfiability Algorithms},
			1999.
			
		\bibitem{PracticalAlgs}
			Joao Marques-Silva,
			 Anton Belov,
			 Ines Lynce.
			\textit{Practical Algorithms for MUS Extraction},
			2012.
			
		\bibitem{MinisatDoc}
			Niklas Eén,
			Niklas Sörensson.
			\textit{An Extensible SAT-solver},
			2003.
			
		
		
	\end{thebibliography}

\end{document}
