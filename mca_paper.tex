\documentclass[]{article}

%opening
\title{Rotation in Minimising Conflicting Assumptions}
\author{Shafik Nassar, Pierre Glianos}

\begin{document}

\maketitle

\begin{abstract}

The \textit{Boolean Satisfiability Problem} (SAT), archetypical NP-Complete, is the problem of determining whether a given boolean formula, usually \textit{CNF}, has a satisfying \textit{assignment}. In addition to the theoretical interest SAT draws, it has many modern-day applications in different fields, most notably hardware verification, with SAT-solvers playing an integral part in Bounded Checking Model, for example.\\
Modern uses require solving SAT instances under given \textit{assumptions}, i.e. partial assignment to the formula's variables. Furthermore, if the instance is \textit{unsatisfiable} under the assumptions, it is interesting to compute a \textit{minimal set of conflicting assumptions}.\\
Any algorithm that computes minimal sets of conflicting assumptions, makes several SAT-solver calls under different assumptions. In this paper, we present algorithms that aim to reduce the number of SAT-solver calls by gathering as much information as possible per call.

\end{abstract}

\pagebreak
\tableofcontents
\pagebreak

\section{Introduction}
\subsection{Introduction to SAT}
\paragraph{Definitions}
\subsection{Introduction to SAT-Solving}
\subsubsection{DPLL}
\subsubsection{CDCL}
\subsection{Minimal set of Conflicting Assumptions}
\subsection[Similar Problem]{Minimizing Conflicting Clauses}

\section{Basic Algorithms}
\subsection{Iterative Insertion}
\subsection{Iterative Deletion}
\subsubsection{Iterative Deletion with Outside Help}

\section{Rotation}
\subsection{Rotation in Conflicting Clauses}
\subsection{Borrowing the Idea}

\section{Implementation}
\subsection{Architecture}

\section{Benchmarks and Results}

\section{Summary}

\end{document}
